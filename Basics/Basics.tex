% preamble: before the document body

% parameters are in []
% example: font size and paper size [12pt, letterpaper]

\documentclass[12pt, a4paper]{article} % declare a type of document (article, book, report, ...)

\usepackage[utf8]{inputenc} % set encoding
\usepackage[T1]{fontenc} % for "đ" and other unicode characters

\usepackage[dvipsnames]{xcolor} % expands the colour pallet; must be declared before many other packages

\usepackage{amsmath} % math package (display mode and complex equations)
\usepackage{hyperref} % hyperlink package

\usepackage{graphicx} % image package
\graphicspath{ {images/} } % image package will look in folder "This-Folder/images" for images
\usepackage{subcaption} % allows multiple images in a figure

\usepackage{float} % "place table exactly here" package

\usepackage[export]{adjustbox} % enables precise size adjustments (textwidth and linewidth)

\usepackage{formular} % create blank underscore lines
\newFRMfield{pagefield}{120mm}
\newFRMfield{longfield}{30mm}

\usepackage{indentfirst} % indent the first paragraph
\setlength{\parindent}{0mm} % indent paragraph
\setlength{\parskip}{2mm} % spacing between paragraphs

\usepackage{setspace}
\onehalfspacing % one and a half spacing in paragraphs

\usepackage{expl3} % repat a command many times
\ExplSyntaxOn
\cs_new_eq:NN \Repeat \prg_replicate:nn
\ExplSyntaxOff



\title{First document}
\author{Hubert Farnsworth}
\date{February 2014} % use the command "\today" to use the compile time date



\begin{document} % start document body

\maketitle % make the preamble elements appear (title, author and date)

% create table of contents based on section, subsection and chapter commands
% add other content with \addcontentsline
\tableofcontents



\begin{abstract}
Special abstract formatting. At the beginning of the document.
\end{abstract}



\section{Numbered section (1 for example)}

\textbf{Bold}

\underline{Underlined}

\textit{Italic}

\textbf{\textit{Bold and Italic}}

{\tiny tiny text}

{\LARGE large text}



Double newline will force a newline and a new paragraph. Not recommended.



\begin{itemize}
  \item Bullet point one in an unordered list
  \item Bullet point two
\end{itemize}

\begin{enumerate}
  \item First item in an ordered list
  \item Second item
\end{enumerate}



\addcontentsline{toc}{section}{Unnumbered Section} % add a section called "Unnumbered Section" to ToC (Table of Contents)
\section*{Unnumbered Section} % * character disables numbering

Text.

\section{Another numbered section (2 for example)}

More text.

\subsection{Numbered subsection (2.1 for example)}

There are a lot more "section"-like commands.



\begin{figure}[H] % image construct
    \centering
    %\includegraphics[max width=0.8\textwidth, max height=0.6\textwidth]{mozilla_logo} % try to display "images/mozilla_logo"
    %\includegraphics[max width=0.8\textwidth, max height=0.6\textwidth]{text.pdf} % include PDF
    \caption{Text below the centered picture}
    \label{fig:mozilla_picture_label}
\end{figure}

Picture number reference: \ref{fig:mozilla_picture_label}

Picture page reference: \pageref{fig:mozilla_picture_label}

% \begin{figure}[h] % multiple images in a figure
%
% \begin{subfigure}{0.5\textwidth}
% \includegraphics[width=0.9\linewidth, left]{image.png}
% \end{subfigure}
%
% \begin{subfigure}{0.5\textwidth}
% \includegraphics[width=0.9\linewidth]{image.png}
% \end{subfigure}
%
% \end{figure}



Inline equation $E=mc^2$ starts and ends with a dollar sign.

\begin{equation} % display mode equation
E=mc^2
\end{equation}

Complex equation:
$$T^{i_1 i_2 \dots i_p}_{j_1 j_2 \dots j_q} = T(x^{i_1},\dots,x^{i_p},e_{j_1},\dots,e_{j_q})$$

$$\int_0^1 \frac{1}{e^x} =  \frac{e-1}{e}$$

$\sin(\beta)$, $\cos(\alpha)$, $\log(x)$


\setlength{\tabcolsep}{20pt} % more space between columns
\renewcommand{\arraystretch}{1.5} % more space between rows

\begin{center}

% H is a "place table exactly here" parameter
% Latex table generator: https://www.tablesgenerator.com/
\begin{table}[H]

\centering

\begin{tabular}{ |l|c|r| } % left, center and right aligned columns; | means separate by a vertical line
\hline                     % add horizontal line to the top of the table
cell1 & cell2 & cell3 \\   % & to break table entry, \\ to go to the next line
cell4 & cell5 & cell6 \\
cell7 & cell8 & cell9 \\
\hline                     % add horizontal line to the bottom of the table
\end{tabular}

\caption{Text below table}
\label{table:sample_table_label}
\end{table}

\end{center}

\end{document} % end document body
